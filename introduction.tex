\section{Introduction}

Multibody System Dynamics (MSD) is a field of study that concentrates on modeling, simulating, and analyzing the dynamics of systems composed of interconnected rigid and flexible bodies. It is a fundamental approach to understanding and predicting the dynamic behavior of complex mechanical systems through sophisticated mathematical models and computational algorithms. The essence of MSD lies in its capacity to accurately describe the motion, forces, and interactions among the components of a system and with the environment, considering both translational and rotational movements and the constraints that govern these movements~\cite{shabana2013dynMSD}. MSD is applicable across various engineering fields, from aerospace and automotive to biomechanics and robotics. This versatility underscores its capability to address the specific dynamics of different systems, enhancing its utility in diverse technological and scientific domains. Also, by accurately predicting system behavior, MSD supports the iterative design and optimization of mechanical systems. Engineers can use MSD to test and refine designs virtually, reducing the need for physical prototypes and accelerating the development process. One important characteristic of MSD is that the detailed models developed using this method make it an excellent foundation for implementing learning-based or model-based control strategies. These strategies, such as Reinforcement Learning (RL)~\cite{sutton_reinforcement_2018} and Model Predictive Control~\cite{schwenzer2021mpc}, can dynamically adapt to the system's behavior, leading to optimized performance even in complex and changing environments.

Reinforcement Learning, a primary control method used in our research, is a branch of Machine Learning (ML) where an agent learns to make decisions by taking actions in an environment to maximize the cumulative reward under acting according to a certain policy~\cite{sutton_reinforcement_2018}. In RL, rewards may not be immediate. An agent may need to perform a series of actions before receiving a reward, creating a delay between the initial action and the eventual reward. The learning process involves exploring various actions and states (exploration) and exploiting the knowledge gained to make better decisions (exploitation). The exploration and exploitation trade-off is one of the crucial concepts in reinforcement learning (RL), which emphasizes the need for a balance that results in efficient agent training. Achieving this balance is essential for the agent to learn effectively and optimize its performance over time. RL can be divided into two main types: model-free methods and model-based methods. The main difference between them is that in model-free methods the agent learns directly from interactions with the environment without trying to understand its dynamics, while in model-based methods the agent learns a model of the environment's dynamics and uses this model to plan and make decisions. Model-free methods are generally simpler and more robust in highly dynamic and complex environments, though they require more interactions for an agent to learn effectively. 
Additionally, RL can be further categorized into traditional Reinforcement Learning (RL) and Deep Reinforcement Learning (DRL). Traditional RL focuses on methods where the agent typically operates with simpler, lower-dimensional state spaces. These methods often rely on tabular representations or linear function approximations. DRL, however, specifically integrates deep neural networks to approximate value functions or policies, so that it can effectively handle high-dimensional state spaces and complex environments. This capability allows DRL to be applied to tasks such as playing video games, robotic control, and autonomous driving~\cite{roveda2020mbrl, mnih2013dqn, zhu2022inverted}. Examples of DRL methods are Advantage-Actor Critic (A2C), Proximal Policy Optimization (PPO) and Deep Q-Network (DQN). A2C is an advanced version of the actor-critic method that uses both policy (actor) and value (critic) functions, with the actor updating the policy based on feedback from the critic, enhancing stability and performance~\cite{mnih2016a2c}. PPO is a popular policy gradient method known for its robustness and simplicity, utilizing a clipped objective function to ensure stable learning~\cite{schulman2017ppo}. DQN combines Q-learning~\cite{sutton_reinforcement_2018} with deep neural networks to handle high-dimensional state spaces, approximating the Q-value function, which represents an expected cumulative reward an agent can achieve by taking a specific action in a given state and following the optimal policy, enabling it to learn effective policies in complex environments~\cite{mnih2013dqn}. 
In Reinforcement Learning, actions (control inputs) can be either discrete or continuous. Discrete actions are suitable for tasks where the set of possible actions is limited and distinct, such as choosing between different predefined options. Continuous actions, on the other hand, are used in scenarios where actions can take any value within a range, such as adjusting the speed of a car or the angle of a robotic arm. 
% RL is inspired by behavioral psychology and is gaining popularity in robotics, game playing, and optimization problems~\cite{roveda2020mbrl, mnih2013dqn, zhu2022inverted}.

The application of RL techniques in the context of MSD is still an emerging field, with limited research available. Current studies have primarily focused on traditional control methods, leaving significant potential for further exploration and development of RL-based approaches to enhance the adaptability and efficiency of MSD models. A noticeable study in this area is done by Kurinov et al.~\cite{kurinov2020autoexcavator}. Researchers have developed an autonomous excavator system utilizing RL as a control base integrated with MSD (detailed simulation with hydraulics and sensors). This study was focused on enabling the excavator to perform various tasks autonomously, including excavation and loading. The effectiveness of their approach was demonstrated in achieving autonomous operation with minimal collisions, showcasing the potential for automation in heavy machinery within the construction and mining industries.  
Additionally, Egli et al.~\cite{egli2022soil} explored the application of reinforcement learning (RL) for adaptive control in hydraulic excavators, aiming to enhance the automation of excavation tasks. Their study introduced a controller capable of adjusting to varying soil properties within a single excavation cycle. Utilizing an RL-based approach, they trained a control policy in simulation, integrating proprioceptive measurements to infer soil characteristics. This method demonstrated significant improvements in adaptability and efficiency compared to traditional excavation techniques. Notably, their controller could operate without explicit soil parameter inputs, relying instead on measurements from hydraulic pressure and kinematic sensors. The experiments conducted on a 12-ton excavator confirmed the controller's ability to maintain high performance across diverse soil conditions, emphasizing the feasibility of RL in real-world excavation scenarios.
Furthermore, Osa and Aizawa~\cite{osa2022deep} investigated deep reinforcement learning with adversarial training for automated excavation. They proposed a novel regularization method that employs virtual adversarial samples to mitigate the overestimation of Q-values in a Q-learning algorithm. Their approach, which integrates depth images for excavation planning, demonstrated superior sample efficiency and robustness in policy learning compared to traditional actor-critic methods~\cite{sutton_reinforcement_2018}. This research highlights the importance of incorporating visual information and advanced regularization techniques to enhance the reliability and effectiveness of RL-based systems.

While RL has shown considerable promise as a control method for complex dynamic systems, it is not without its limitations. One of the primary challenges of RL is its high demand for extensive interactions with the environment to learn effective policies, which can result in long training times and high computational costs. Additionally, RL algorithms often struggle with the exploration-exploitation trade-off, especially in high-dimensional state spaces and non-linear dynamics typical of MSD. These factors can make RL less efficient and robust when applied directly to complex mechanical systems~\cite{dulac2019challenges}.
This is where Curriculum Learning (CL) can provide substantial benefits. CL is about structuring training content and learning experiences in a way that mirrors the progression from simple to complex, akin to how a curriculum in educational settings is designed~\cite{bengio2009cl}. This method may not only accelerate the learning process but also enhance the robustness and adaptability of the control strategies.
Research by Hacohen et al.~\cite{hacohen2019clnetworks} delves into the effectiveness of CL in the training of deep neural networks, with a focus on how CL can significantly impact the efficiency of the learning process. Experiments to analyze the impact of curriculum-based learning on the training efficiency of deep neural networks were conducted and the curriculum was structured around a series of tasks, each representing a different level of difficulty, designed to challenge the neural network progressively. This method allowed for the examination of how neural networks adapt their learning strategies in response to escalating task complexity. \hl{This work suggests that by implementing a curriculum-based approach deep RL algorithms can develop more robust and effective strategies.}\todo[color=green]{Remember to check the robustness of our CRL examples.}


The work by Narvekar et al.~\cite{narvekar2020survey} is particularly instructive for our research as it provides a taxonomy of CL methodologies that can be directly applied to the RL challenges we could face in tandem with MSD. They categorize CL strategies into three main approaches: task sequencing, transfer learning, and multi-task learning. Task sequencing involves arranging learning tasks in a meaningful order to gradually increase complexity, thereby enhancing the learning efficiency and robustness of RL agents. Transfer learning focuses on leveraging knowledge gained from previous tasks to improve performance on new, related tasks. Multi-task learning allows simultaneous training on multiple tasks to share knowledge and improve overall learning efficiency. \hl{In our context, task sequencing is particularly applicable.}\todo{Is this our method of choice? If yes, you should replace discussion on TL with task sequencing examples.} By structuring the learning experiences from simple to complex tasks, we can systematically improve the RL agent's ability to handle the difficulties of mechanical systems in MSD. This systematic enhancement of learning experiences can mitigate some of the efficiency and robustness issues traditionally associated with RL in dynamic, high-dimensional environments.

The study written by Gupta et al.~\cite{gupta2021rlcapabilities} investigates how integrating CL can enhance RL for continuous control tasks, which are highly relevant to MSD. Their research focuses on tasks such as robotic arm manipulation and locomotion, demonstrating that various CL strategies, like progressive task sequencing and parameter adaptation, significantly improve learning efficiency and performance. 

The process of knowledge transfer within CL can be effectively combined with Transfer Learning (TL) to enhance learning efficiency.\todo{TL was already discussed and you should introduce it earlier. Also, why to discuss TL in such details if we are not using it directly?}
Transfer learning focuses on leveraging knowledge gained from previous tasks to improve performance on new, related tasks. As written by Taylor et al.~\cite{taylor2009transferlearning}, TL emphasizes the importance of prior learning in tackling subsequent, more complex tasks. This approach is directly applicable to the nature of control tasks in MSD, where knowledge from simpler tasks can be transferred to more complex ones, enhancing the overall learning process. Combining CL and TL can thus provide a structured and efficient way to develop robust RL agents for complex mechanical systems. Furthermore, Weinshall's work~\cite{weinshall2018cltransfer} have delved into the integration of TL within CL, highlighting how accumulated knowledge can significantly boost performance in more complex systems. Their work demonstrates that using a pre-trained neural network on a different task to guide the curriculum can approximate an ideal learning sequence, leading to faster convergence and improved generalization. 
A paper by Bhati et al.~\cite{bhati2023clmulti} demonstrates the effectiveness of CL in a multi-agent RL environment. A sequential task structure was implemented where each task increased in complexity and inter-agent dependency. This approach was designed to incrementally develop and refine the cooperative strategies of individual agents within a simulated multi-agent environment, allowing researchers to observe how agents adapted and optimized their performance in progressively challenging scenarios. This finding is crucial for tasks where teamwork among agents is a key factor. 

\hl{The perspectives of using Curriculum Reinforcement Learning (CRL) in MSD are promising. CRL, which integrates CL with RL, offers a structured approach to tackle the complexities of MSD by gradually increasing the difficulty of control tasks. This methodology not only enhances learning efficiency but also improves the adaptability and robustness of RL agents in managing dynamic mechanical systems.}\todo{Here, you should in addition discuss the continous RL as well. Now, the continous RL in later part of the paragraph is out of the context.} In our prior work by Manzl et al.~\cite{manzl2023relrl}, we evaluated RL algorithms for controlling mechanical systems using discrete control methods, effectively stabilizing single to triple link inverted pendulum systems. The RL approach demonstrated adaptability and effectiveness in managing the dynamics and complexities of these models. Our current study extends this work by applying a novel continuous control method combined with CL technique. Using the same multi-link inverted pendulum on a cart system, we can clearly demonstrate the improvements afforded by these advanced methods. By switching from discrete to continuous action space, we enable smoother and more precise adjustments in system behavior with the faster training time, which are critical when managing the complex dynamics of multibody systems. Moreover, integrating CL accelerates the training process more, allowing RL agents to quickly adapt to increasingly complex tasks. This dual approach not only streamlines control operations but also significantly enhances learning efficiency. Our empirical evidence demonstrates the effectiveness of these strategies in a controlled mechanical environment, highlighting their practical implications. This research paves the way for developing more advanced, autonomous control systems that could transform interactions with mechanical systems across various industries.
% In our prior work by Manzl et al.~\cite{manzl2023relrl}, we evaluated RL algorithms for controlling mechanical systems using discrete control methods, effectively stabilizing single to triple link inverted pendulum systems. The RL approach demonstrated adaptability and effectiveness in managing the dynamics and complexities of these models. Our current study extends the previous one with a novel application of continuous control strategy combined with Curriculum Learning techniques. As a mechanical model we use the same multi-link inverted pendulum on a cart system, so that the improvements can be properly shown. By adopting continuous control, we enable smoother and more precise adjustments in system behavior, which are critical when managing the intricate dynamics of multibody systems. Moreover, by integrating CL, we accelerate the training process, allowing RL agents to quickly adapt to increasingly complex tasks. This dual approach not only streamlines control operations but also significantly enhances learning efficiency. Our empirical evidence demonstrates the effectiveness of these strategies in a controlled mechanical environment, highlighting their practical implications. The successful application of continuous control alongside CL in RL for MSD not only advances our understanding of sophisticated control strategies for complex systems but also establishes a foundation for future explorations in this field. This research paves the way for developing more advanced, autonomous control systems that could transform interactions with mechanical systems across various industries.